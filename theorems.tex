% ============================================================================
% THEOREM-LIKE ENVIRONMENTS
% ============================================================================
% Definitions for all theorem-like structures used in the dissertation.
% Requires: amsthm package

% Verify amsthm package is loaded
\makeatletter
\@ifpackageloaded{amsthm}{}{%
  \PackageError{theorems}{amsthm package required}{%
    This file requires the amsthm package to be loaded first.%
  }%
}
\makeatother

% Plain style (italic body) for mathematical statements
\theoremstyle{plain}
\newtheorem{theorem}{Theorem}[section]
\newtheorem{lemma}[theorem]{Lemma}
\newtheorem{corollary}[theorem]{Corollary}
\newtheorem{proposition}[theorem]{Proposition}
\newtheorem{hypothesis}[theorem]{Hypothesis}

% Definition style (upright body) for definitions and descriptive content
\theoremstyle{definition}
\newtheorem{axiom}{Axiom}
\newtheorem{definition}{Definition}[section]
\newtheorem{principle}{Principle}
\newtheorem{directive}{Directive}
\newtheorem{example}{Example}[section]
\newtheorem{problem}{Problem}[section]
\newtheorem{pattern}{Pattern}[section]
\newtheorem{design-decision}{Design Decision}[section]
\newtheorem{fallacy}{Fallacy}[section]
\newtheorem{observation}{Observation}[section]
\newtheorem{instantiation}{Instantiation}[section]
\newtheorem{construction}{Construction}[section]
\newtheorem{speculation}{Speculation}[section]

% Remark style for notes and remarks
\theoremstyle{remark}
\newtheorem{remark}{Remark}
