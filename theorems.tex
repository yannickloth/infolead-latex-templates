% ============================================================================
% UNIFIED THEOREM-LIKE ENVIRONMENTS
% ============================================================================
% Comprehensive definitions for all theorem-like structures used in academic
% and scientific documents. Combines standard mathematical theorems,
% IVP-specific environments, and scientific claim environments with enhanced
% visual styling.
%
% Requires: amsthm, tcolorbox, xcolor packages
%
% This file supersedes:
%   - theorems-standard.tex (generic math theorems)
%   - theorems-ivp.tex (IVP-specific theorems)
%   - theorems-scientific.tex (scientific claims with tcolorbox styling)
%
% Styling uses subtle semantic colors with distinct border patterns for B&W print compatibility

% Verify required packages are loaded
\makeatletter
\@ifpackageloaded{amsthm}{}{%
  \PackageError{theorems}{amsthm package required}{%
    This file requires the amsthm package to be loaded first.%
  }%
}
\@ifpackageloaded{tcolorbox}{}{%
  \usepackage[most]{tcolorbox}%
}
\makeatother

% Load tcolorbox theorem library for enhanced boxes
\tcbuselibrary{theorems,skins,breakable}

% ============================================================================
% PART 1: STANDARD MATHEMATICAL ENVIRONMENTS
% ============================================================================
% Generic mathematical theorem-like structures commonly used in academic writing.

% Plain style (italic body) for mathematical statements
\theoremstyle{plain}
\newtheorem{theorem}{Theorem}[chapter]
\newtheorem{lemma}[theorem]{Lemma}
\newtheorem{corollary}[theorem]{Corollary}
\newtheorem{proposition}[theorem]{Proposition}

% Definition style (upright body) for definitions
\theoremstyle{definition}
\newtheorem{definition}{Definition}[chapter]
\newtheorem{example}{Example}[chapter]

% Remark style for notes and remarks
% Custom environment with closing marker for multi-paragraph clarity
\theoremstyle{remark}
\newtheorem{remark@inner}{Remark}
\newenvironment{remark}
  {\begin{remark@inner}}
  {\hfill$\diamond$\end{remark@inner}}

% ============================================================================
% PART 2: IVP AND DESIGN THEORY ENVIRONMENTS
% ============================================================================
% Custom theorem-like structures for IVP (Independent Variation Principle)
% and software design theory documents.

\theoremstyle{definition}
\newtheorem{principle}{Principle}
\newtheorem{directive}{Directive}
\newtheorem{problem}{Problem}
\newtheorem{pattern}{Pattern}
\newtheorem{design-decision}{Design Decision}
\newtheorem{fallacy}{Fallacy}
\newtheorem{observation}{Observation}
\newtheorem{instantiation}{Instantiation}
\newtheorem{construction}{Construction}
\newtheorem{speculation}{Speculation}

% ============================================================================
% PART 3: SCIENTIFIC CLAIM ENVIRONMENTS (with tcolorbox styling)
% ============================================================================
% Custom theorem-like structures for scientific manuscripts that need to
% distinguish between different epistemic statuses of claims.
% Styling uses subtle semantic colors with icons for quick identification,
% plus distinct border patterns that remain distinguishable in B&W print.

% ----------------------------------------------------------------------------
% COLOR DEFINITIONS (subtle semantic colors)
% ----------------------------------------------------------------------------
% Each environment type has a distinct but subtle color to aid quick identification.
% Colors are chosen to be subtle enough for academic reading while providing
% semantic distinction. Border patterns (solid, dashed, dotted, etc.) remain
% as secondary visual cues for B&W printing.

% Achievement: Subtle green tint (success/breakthrough)
\definecolor{achievementbg}{RGB}{240, 250, 240}       % Very light green
\definecolor{achievementframe}{RGB}{46, 125, 50}      % Forest green

% Prediction: Subtle blue tint (future/unknown)
\definecolor{predictionbg}{RGB}{240, 248, 255}        % Alice blue
\definecolor{predictionframe}{RGB}{30, 90, 150}       % Steel blue

% Postdiction: Subtle blue-gray tint (verified retrospective)
\definecolor{postdictionbg}{RGB}{245, 248, 250}       % Very light slate
\definecolor{postdictionframe}{RGB}{70, 100, 130}     % Slate blue

% Warning: Subtle amber/orange tint (caution)
\definecolor{warningbg}{RGB}{255, 250, 240}           % Floral white/cream
\definecolor{warningframe}{RGB}{180, 95, 6}           % Dark orange

% Open Question: Subtle purple tint (unresolved/mystery)
\definecolor{openquestionbg}{RGB}{248, 245, 255}      % Lavender tint
\definecolor{openquestionframe}{RGB}{106, 90, 150}    % Muted purple

% Requirement: Subtle red tint (mandatory/critical)
\definecolor{requirementbg}{RGB}{255, 245, 245}       % Very light pink
\definecolor{requirementframe}{RGB}{150, 50, 50}      % Dark red

% Hypothesis: Subtle yellow tint (tentative/uncertain)
\definecolor{hypothesisbg}{RGB}{255, 252, 240}        % Very light yellow
\definecolor{hypothesisframe}{RGB}{160, 130, 40}      % Dark gold

% Axiom: Subtle deep blue tint (foundational/bedrock)
\definecolor{axiombg}{RGB}{240, 245, 255}             % Very light blue
\definecolor{axiomframe}{RGB}{25, 50, 100}            % Navy blue

% Assumption: Subtle gray-blue tint (conditional/bracketed)
\definecolor{assumptionbg}{RGB}{248, 250, 252}        % Very light slate gray
\definecolor{assumptionframe}{RGB}{80, 100, 120}      % Slate gray

% Consistency Check: Subtle teal tint (verification/check)
\definecolor{consistencybg}{RGB}{240, 252, 250}       % Very light teal
\definecolor{consistencyframe}{RGB}{40, 110, 100}     % Teal

% ----------------------------------------------------------------------------
% ICON DEFINITIONS
% ----------------------------------------------------------------------------
% Icons provide quick visual identification of environment types.
% Using standard LaTeX symbols for maximum compatibility.

\newcommand{\achievementicon}{\ensuremath{\bigstar}}           % Star for breakthroughs
\newcommand{\predictionicon}{\ensuremath{\rightarrow}}         % Arrow for future
\newcommand{\postdictionicon}{\ensuremath{\checkmark}}         % Check for verified
\newcommand{\warningicon}{\ensuremath{\triangle}}              % Triangle for caution
\newcommand{\openquestionicon}{\textbf{?}}                     % Question mark
\newcommand{\requirementicon}{\ensuremath{\bullet}}            % Bullet for mandatory
\newcommand{\hypothesisicon}{\ensuremath{\sim}}                % Tilde for tentative
\newcommand{\axiomicon}{\ensuremath{\blacksquare}}             % Solid square for foundation
\newcommand{\assumptionicon}{\ensuremath{[\,]}}                % Brackets for conditional
\newcommand{\consistencyicon}{\ensuremath{\equiv}}             % Equivalence for verification

% ----------------------------------------------------------------------------
% COUNTER SETUP
% ----------------------------------------------------------------------------
% Shared counter for all scientific claim environments (numbered sequentially)
\newcounter{scientificclaim}[chapter]
\renewcommand{\thescientificclaim}{\thechapter.\arabic{scientificclaim}}

% ----------------------------------------------------------------------------
% PAGE BREAK BEHAVIOR
% ----------------------------------------------------------------------------
% All scientific claim boxes use 'breakable' but with settings that keep them
% together unless they genuinely exceed the available page space. The box will
% be moved to the next page if it doesn't fit, and only break across pages if
% it's longer than a full page.

% ----------------------------------------------------------------------------
% ACHIEVEMENT ENVIRONMENT
% ----------------------------------------------------------------------------
% For novel findings and breakthrough results unique to the theory.
% Styled with double-line border for maximum visibility in B&W.

\newtcolorbox[use counter=scientificclaim, number within=chapter]{achievement}[1][]{%
  enhanced,
  breakable,
  break at=-\baselineskip,
  colback=achievementbg,
  colframe=achievementframe,
  boxrule=0.8pt,
  arc=2pt,
  left=8pt,
  right=8pt,
  top=6pt,
  bottom=6pt,
  fonttitle=\bfseries\sffamily,
  title={\achievementicon~Achievement~\thescientificclaim\ifstrempty{#1}{}{: #1}},
  before upper={\parindent=1.5em},
  borderline={1.5pt}{-3pt}{achievementframe!50},
  coltitle=black,
  attach boxed title to top left={yshift=-2mm, xshift=4mm},
  boxed title style={
    colback=achievementbg,
    colframe=achievementframe,
    boxrule=0.5pt,
    arc=1pt
  }
}

% ----------------------------------------------------------------------------
% PREDICTION ENVIRONMENT
% ----------------------------------------------------------------------------
% For testable, falsifiable predictions.
% Styled with dashed border to indicate "future verification needed".

\newtcolorbox[use counter=scientificclaim, number within=chapter]{prediction}[1][]{%
  enhanced,
  breakable,
  break at=-\baselineskip,
  colback=predictionbg,
  colframe=predictionframe,
  boxrule=1pt,
  arc=0pt,
  left=8pt,
  right=8pt,
  top=6pt,
  bottom=6pt,
  fonttitle=\bfseries\sffamily,
  title={\predictionicon~Prediction~\thescientificclaim\ifstrempty{#1}{}{: #1}},
  before upper={\parindent=1.5em},
  borderline={0.8pt}{0pt}{predictionframe, dashed},
  coltitle=black,
  attach boxed title to top left={yshift=-2mm, xshift=4mm},
  boxed title style={
    colback=predictionbg,
    colframe=predictionframe,
    boxrule=0.5pt,
    arc=0pt
  }
}

% ----------------------------------------------------------------------------
% WARNING ENVIRONMENT
% ----------------------------------------------------------------------------
% For caveats, limitations, and scope boundaries.
% Styled with left bar for quick visual identification.

\newtcolorbox[use counter=scientificclaim, number within=chapter]{warning}[1][]{%
  enhanced,
  breakable,
  break at=-\baselineskip,
  colback=warningbg,
  colframe=warningframe,
  boxrule=0pt,
  arc=0pt,
  left=10pt,
  right=8pt,
  top=6pt,
  bottom=6pt,
  fonttitle=\bfseries\sffamily,
  title={\warningicon~Warning~\thescientificclaim\ifstrempty{#1}{}{: #1}},
  before upper={\parindent=1.5em},
  borderline west={3pt}{0pt}{warningframe},
  coltitle=black,
  attach boxed title to top left={yshift=-2mm, xshift=4mm},
  boxed title style={
    colback=warningbg,
    colframe=warningframe,
    boxrule=0.5pt,
    arc=0pt
  }
}

% ----------------------------------------------------------------------------
% OPEN QUESTION ENVIRONMENT
% ----------------------------------------------------------------------------
% For unresolved problems and directions for future research.
% Styled with dotted border to indicate "incomplete/open".

\newtcolorbox[use counter=scientificclaim, number within=chapter]{open_question}[1][]{%
  enhanced,
  breakable,
  break at=-\baselineskip,
  colback=openquestionbg,
  colframe=openquestionframe,
  boxrule=0pt,
  arc=2pt,
  left=8pt,
  right=8pt,
  top=6pt,
  bottom=6pt,
  fonttitle=\bfseries\sffamily,
  title={\openquestionicon~Open Question~\thescientificclaim\ifstrempty{#1}{}{: #1}},
  before upper={\parindent=1.5em},
  borderline={1pt}{0pt}{openquestionframe, dotted},
  coltitle=black,
  attach boxed title to top left={yshift=-2mm, xshift=4mm},
  boxed title style={
    colback=openquestionbg,
    colframe=openquestionframe,
    boxrule=0.5pt,
    arc=1pt
  }
}

% ----------------------------------------------------------------------------
% REQUIREMENT ENVIRONMENT
% ----------------------------------------------------------------------------
% For necessary conditions and consistency requirements.
% Styled with solid border and corner markers for "must have" emphasis.

\newtcolorbox[use counter=scientificclaim, number within=chapter]{requirement}[1][]{%
  enhanced,
  breakable,
  break at=-\baselineskip,
  colback=requirementbg,
  colframe=requirementframe,
  boxrule=0.6pt,
  arc=0pt,
  left=8pt,
  right=8pt,
  top=6pt,
  bottom=6pt,
  fonttitle=\bfseries\sffamily,
  title={\requirementicon~Requirement~\thescientificclaim\ifstrempty{#1}{}{: #1}},
  before upper={\parindent=1.5em},
  overlay={
    \draw[requirementframe, line width=1.5pt]
      ([xshift=3pt,yshift=-3pt]frame.north west) -- ++(8pt,0)
      ([xshift=3pt,yshift=-3pt]frame.north west) -- ++(0,-8pt);
    \draw[requirementframe, line width=1.5pt]
      ([xshift=-3pt,yshift=-3pt]frame.north east) -- ++(-8pt,0)
      ([xshift=-3pt,yshift=-3pt]frame.north east) -- ++(0,-8pt);
    \draw[requirementframe, line width=1.5pt]
      ([xshift=3pt,yshift=3pt]frame.south west) -- ++(8pt,0)
      ([xshift=3pt,yshift=3pt]frame.south west) -- ++(0,8pt);
    \draw[requirementframe, line width=1.5pt]
      ([xshift=-3pt,yshift=3pt]frame.south east) -- ++(-8pt,0)
      ([xshift=-3pt,yshift=3pt]frame.south east) -- ++(0,8pt);
  },
  coltitle=black,
  attach boxed title to top left={yshift=-2mm, xshift=4mm},
  boxed title style={
    colback=requirementbg,
    colframe=requirementframe,
    boxrule=0.5pt,
    arc=0pt
  }
}

% ----------------------------------------------------------------------------
% POSTDICTION ENVIRONMENT
% ----------------------------------------------------------------------------
% For derived results that match already-known data (epistemic honesty).
% Styled similarly to prediction but with solid border (already verified).

\newtcolorbox[use counter=scientificclaim, number within=chapter]{postdiction}[1][]{%
  enhanced,
  breakable,
  break at=-\baselineskip,
  colback=postdictionbg,
  colframe=postdictionframe,
  boxrule=1pt,
  arc=0pt,
  left=8pt,
  right=8pt,
  top=6pt,
  bottom=6pt,
  fonttitle=\bfseries\sffamily,
  title={\postdictionicon~Postdiction~\thescientificclaim\ifstrempty{#1}{}{: #1}},
  before upper={\parindent=1.5em},
  coltitle=black,
  attach boxed title to top left={yshift=-2mm, xshift=4mm},
  boxed title style={
    colback=postdictionbg,
    colframe=postdictionframe,
    boxrule=0.5pt,
    arc=0pt
  }
}

% ----------------------------------------------------------------------------
% CONSISTENCY CHECK ENVIRONMENT
% ----------------------------------------------------------------------------
% For verification that the theory reproduces known physics.

\newtcolorbox[use counter=scientificclaim, number within=chapter]{consistency_check}[1][]{%
  enhanced,
  breakable,
  break at=-\baselineskip,
  colback=consistencybg,
  colframe=consistencyframe,
  boxrule=0.5pt,
  arc=2pt,
  left=8pt,
  right=8pt,
  top=6pt,
  bottom=6pt,
  fonttitle=\bfseries\sffamily,
  title={\consistencyicon~Consistency Check~\thescientificclaim\ifstrempty{#1}{}{: #1}},
  before upper={\parindent=1.5em},
  coltitle=black,
  attach boxed title to top left={yshift=-2mm, xshift=4mm},
  boxed title style={
    colback=consistencybg,
    colframe=consistencyframe,
    boxrule=0.5pt,
    arc=1pt
  }
}

% ----------------------------------------------------------------------------
% HYPOTHESIS ENVIRONMENT
% ----------------------------------------------------------------------------
% For unproven assumptions and working hypotheses that require validation.
% Styled with dash-dot border to indicate "tentative/uncertain".

\newtcolorbox[use counter=scientificclaim, number within=chapter]{hypothesis}[1][]{%
  enhanced,
  breakable,
  break at=-\baselineskip,
  colback=hypothesisbg,
  colframe=hypothesisframe,
  boxrule=0pt,
  arc=2pt,
  left=8pt,
  right=8pt,
  top=6pt,
  bottom=6pt,
  fonttitle=\bfseries\sffamily,
  title={\hypothesisicon~Hypothesis~\thescientificclaim\ifstrempty{#1}{}{: #1}},
  before upper={\parindent=1.5em},
  borderline={1pt}{0pt}{hypothesisframe, dash dot},
  coltitle=black,
  attach boxed title to top left={yshift=-2mm, xshift=4mm},
  boxed title style={
    colback=hypothesisbg,
    colframe=hypothesisframe,
    boxrule=0.5pt,
    arc=1pt
  }
}

% ----------------------------------------------------------------------------
% AXIOM ENVIRONMENT
% ----------------------------------------------------------------------------
% For foundational statements that are taken as given.
% Styled with thick double border for maximum prominence (foundational claims).

\newtcolorbox[use counter=scientificclaim, number within=chapter]{axiom}[1][]{%
  enhanced,
  breakable,
  break at=-\baselineskip,
  colback=axiombg,
  colframe=axiomframe,
  boxrule=1.2pt,
  arc=0pt,
  left=8pt,
  right=8pt,
  top=6pt,
  bottom=6pt,
  fonttitle=\bfseries\sffamily,
  title={\axiomicon~Axiom~\thescientificclaim\ifstrempty{#1}{}{: #1}},
  before upper={\parindent=1.5em},
  borderline={2pt}{-4pt}{axiomframe},
  coltitle=black,
  attach boxed title to top left={yshift=-2mm, xshift=4mm},
  boxed title style={
    colback=axiombg,
    colframe=axiomframe,
    boxrule=0.8pt,
    arc=0pt
  }
}

% ----------------------------------------------------------------------------
% ASSUMPTION ENVIRONMENT
% ----------------------------------------------------------------------------
% For working assumptions with similar epistemic status to hypothesis.
% Styled with left and right bars to indicate "bracketed/conditional".

\newtcolorbox[use counter=scientificclaim, number within=chapter]{assumption}[1][]{%
  enhanced,
  breakable,
  break at=-\baselineskip,
  colback=assumptionbg,
  colframe=assumptionframe,
  boxrule=0pt,
  arc=0pt,
  left=10pt,
  right=10pt,
  top=6pt,
  bottom=6pt,
  fonttitle=\bfseries\sffamily,
  title={\assumptionicon~Assumption~\thescientificclaim\ifstrempty{#1}{}{: #1}},
  before upper={\parindent=1.5em},
  borderline west={2pt}{0pt}{assumptionframe},
  borderline east={2pt}{0pt}{assumptionframe},
  coltitle=black,
  attach boxed title to top left={yshift=-2mm, xshift=4mm},
  boxed title style={
    colback=assumptionbg,
    colframe=assumptionframe,
    boxrule=0.5pt,
    arc=0pt
  }
}

% ----------------------------------------------------------------------------
% CONCLUSION ENVIRONMENT (section-level)
% ----------------------------------------------------------------------------
% For chapter/section conclusions summarizing key points.

\theoremstyle{remark}
\newtheorem{conclusion}{Conclusion}

% ----------------------------------------------------------------------------
% DERIVATION ENVIRONMENT
% ----------------------------------------------------------------------------
% For step-by-step mathematical derivations that aren't formal theorems.

\theoremstyle{definition}
\newtheorem{derivation@inner}[scientificclaim]{Derivation}
\newenvironment{derivation}[1][]
  {\begin{derivation@inner}\ifstrempty{#1}{}{\textnormal{(#1)}}\itshape}
  {\hfill$\square$\end{derivation@inner}}

% ============================================================================
% STARRED VARIANTS (unnumbered)
% ============================================================================
% For environments that don't need numbering (e.g., in appendices)

\newtcolorbox{achievement*}[1][]{%
  enhanced,
  breakable,
  break at=-\baselineskip,
  colback=achievementbg,
  colframe=achievementframe,
  boxrule=0.8pt,
  arc=2pt,
  left=8pt,
  right=8pt,
  top=6pt,
  bottom=6pt,
  fonttitle=\bfseries\sffamily,
  title={\achievementicon~Achievement\ifstrempty{#1}{}{: #1}},
  before upper={\parindent=1.5em},
  borderline={1.5pt}{-3pt}{achievementframe!50},
  coltitle=black,
  attach boxed title to top left={yshift=-2mm, xshift=4mm},
  boxed title style={
    colback=achievementbg,
    colframe=achievementframe,
    boxrule=0.5pt,
    arc=1pt
  }
}

\newtcolorbox{prediction*}[1][]{%
  enhanced,
  breakable,
  break at=-\baselineskip,
  colback=predictionbg,
  colframe=predictionframe,
  boxrule=1pt,
  arc=0pt,
  left=8pt,
  right=8pt,
  top=6pt,
  bottom=6pt,
  fonttitle=\bfseries\sffamily,
  title={\predictionicon~Prediction\ifstrempty{#1}{}{: #1}},
  before upper={\parindent=1.5em},
  borderline={0.8pt}{0pt}{predictionframe, dashed},
  coltitle=black,
  attach boxed title to top left={yshift=-2mm, xshift=4mm},
  boxed title style={
    colback=predictionbg,
    colframe=predictionframe,
    boxrule=0.5pt,
    arc=0pt
  }
}

\newtcolorbox{open_question*}[1][]{%
  enhanced,
  breakable,
  break at=-\baselineskip,
  colback=openquestionbg,
  colframe=openquestionframe,
  boxrule=0pt,
  arc=2pt,
  left=8pt,
  right=8pt,
  top=6pt,
  bottom=6pt,
  fonttitle=\bfseries\sffamily,
  title={\openquestionicon~Open Question\ifstrempty{#1}{}{: #1}},
  before upper={\parindent=1.5em},
  borderline={1pt}{0pt}{openquestionframe, dotted},
  coltitle=black,
  attach boxed title to top left={yshift=-2mm, xshift=4mm},
  boxed title style={
    colback=openquestionbg,
    colframe=openquestionframe,
    boxrule=0.5pt,
    arc=1pt
  }
}

\newtcolorbox{warning*}[1][]{%
  enhanced,
  breakable,
  break at=-\baselineskip,
  colback=warningbg,
  colframe=warningframe,
  boxrule=0pt,
  arc=0pt,
  left=10pt,
  right=8pt,
  top=6pt,
  bottom=6pt,
  fonttitle=\bfseries\sffamily,
  title={\warningicon~Warning\ifstrempty{#1}{}{: #1}},
  before upper={\parindent=1.5em},
  borderline west={3pt}{0pt}{warningframe},
  coltitle=black,
  attach boxed title to top left={yshift=-2mm, xshift=4mm},
  boxed title style={
    colback=warningbg,
    colframe=warningframe,
    boxrule=0.5pt,
    arc=0pt
  }
}

\newtcolorbox{hypothesis*}[1][]{%
  enhanced,
  breakable,
  break at=-\baselineskip,
  colback=hypothesisbg,
  colframe=hypothesisframe,
  boxrule=0pt,
  arc=2pt,
  left=8pt,
  right=8pt,
  top=6pt,
  bottom=6pt,
  fonttitle=\bfseries\sffamily,
  title={\hypothesisicon~Hypothesis\ifstrempty{#1}{}{: #1}},
  before upper={\parindent=1.5em},
  borderline={1pt}{0pt}{hypothesisframe, dash dot},
  coltitle=black,
  attach boxed title to top left={yshift=-2mm, xshift=4mm},
  boxed title style={
    colback=hypothesisbg,
    colframe=hypothesisframe,
    boxrule=0.5pt,
    arc=1pt
  }
}

\newtcolorbox{axiom*}[1][]{%
  enhanced,
  breakable,
  break at=-\baselineskip,
  colback=axiombg,
  colframe=axiomframe,
  boxrule=1.2pt,
  arc=0pt,
  left=8pt,
  right=8pt,
  top=6pt,
  bottom=6pt,
  fonttitle=\bfseries\sffamily,
  title={\axiomicon~Axiom\ifstrempty{#1}{}{: #1}},
  before upper={\parindent=1.5em},
  borderline={2pt}{-4pt}{axiomframe},
  coltitle=black,
  attach boxed title to top left={yshift=-2mm, xshift=4mm},
  boxed title style={
    colback=axiombg,
    colframe=axiomframe,
    boxrule=0.8pt,
    arc=0pt
  }
}

\newtcolorbox{assumption*}[1][]{%
  enhanced,
  breakable,
  break at=-\baselineskip,
  colback=assumptionbg,
  colframe=assumptionframe,
  boxrule=0pt,
  arc=0pt,
  left=10pt,
  right=10pt,
  top=6pt,
  bottom=6pt,
  fonttitle=\bfseries\sffamily,
  title={\assumptionicon~Assumption\ifstrempty{#1}{}{: #1}},
  before upper={\parindent=1.5em},
  borderline west={2pt}{0pt}{assumptionframe},
  borderline east={2pt}{0pt}{assumptionframe},
  coltitle=black,
  attach boxed title to top left={yshift=-2mm, xshift=4mm},
  boxed title style={
    colback=assumptionbg,
    colframe=assumptionframe,
    boxrule=0.5pt,
    arc=0pt
  }
}

% ============================================================================
% CHAPTER ABSTRACT ENVIRONMENT
% ============================================================================
% For chapter-level summaries that should not create a new chapter
% (unlike the document-level abstract environment)

\newenvironment{chapterabstract}{%
  \par\vspace{\baselineskip}%
  \begin{quotation}%
  \noindent\textit{\textbf{Abstract:}}\quad%
}{%
  \end{quotation}%
  \par\vspace{\baselineskip}%
}

% ============================================================================
% QUESTIONS AND ANSWERS (Q&A) ENVIRONMENT
% ============================================================================
% For chapter-level Q&A sections that clarify concepts and explore consequences.
% Each chapter resets the counter using \setcounter{qacounter}{0}.

\newcounter{qacounter}
\newcommand{\qa}[1]{\refstepcounter{qacounter}\subsection*{Q\theqacounter: #1}}
